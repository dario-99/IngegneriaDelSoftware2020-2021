\section{Analisi e specifica dei requisiti}
Introduciamo adesso l'analisi e la specifica dei requisiti,
\`e generalmente la prima fase del processo di sviluppo software,
ed \`e anche la fase pi\`u importante.
In questa fase dobbiamo definire in modo formale, i requisiti della nostra applicazione.
Errori in questa fase costano molto caro, infatti si propagano in tutte le altre fasi,
con perdita di tempo e denaro.
\\
Le domande principali da farci sono:
\begin{itemize}
    \item Cosa sono i requisiti?
    \item Che cosa produce questa fase?
\end{itemize}
in questo capitolo troveremo le risposte a queste domande.
\subsection{Ingegneria dei requisiti}
l'ingengeria dei requisiti (requirements engineering), si occupa di \textbf{raccogliere
, documentare, classificare, analizzare e gestire i requisiti}, cio\`e le funzionalit\`a, le 
caratteristiche e i vincoli richiesti al nostro sistema.

L'Ingegneria dei requisiti \`e di fondamentale importanza, come abbiamo detto errori
nella fase della specifica di questi requisiti, si propagano.
E senza un approccio \textbf{chiaro e sistemico}, \`e facile comprendere male 
le richieste con la conseguenza di:
\begin{itemize}
    \item Clienti non soddisfatti
    \item Ritardi nella produzione
    \item Aumento dei costi di manutenzione
    \item Software che non rispetta le richieste del cliente
\end{itemize}
\subsubsection{I requisiti}
I requisiti sono:
\begin{quote}
    una frase in linguaggio naturale che descrive qualcosa che il sistema dovr\`a
    fare \textbf{(requisito funzionale)}, o una propriet\`a o vincolo che si desidera per il sistema
    \textbf{(requisito non funzionale)}, e che uno o pi\`u stakeholders (portatori di interesse) 
    richiedono al sistema stesso
\end{quote}
Sintetizzando un requisito, esprime a parole, cosa il sistema dovr\`a fare, le caratteristiche e i vincoli.
\subsubsection{Tipi di requisiti}
I requisiti si dividono in base alla \textbf{astrazione} e al \textbf{tipo}.
\begin{center}
    \textbf{Astrazione}
    \begin{itemize}
        \item I requisiti \textbf{utente}, sono requisiti pi\`u astratti che descrivono in modo approssimativo
                le funzionalit\`a del sistema.
        \item i requisiti \textbf{di sistema}, sono meno astratti e vanno a descrivere in modo 
                approfondito, le singole funzionalit\`a.
    \end{itemize}    
    \textbf{Tipo}
    \begin{itemize}
        \item I requisiti \textbf{funzionali}, riguardano i vincoli e le funzionalit\`a richieste
                al sistema.
        \item i requisiti \textbf{non funzionali}, riguardano gli attributi di qualit\`a che
                il sistema deve soddisfare, ad esempio il cliente pu\`o chiedere un software facile da usare
                e sar\`a poi compito del analista di quantificare questo requisito.
                Ma i requisiti non funzionali riguardano anche i vincoli di sviluppo, tecnologici.
                (ad esempio un vincolo sul linguaggio di programmazione da usare)
        \item i requisiti \textbf{di dominio}, sono vincoli e richieste che sono proprie del dominio
                applicativo in cui agir\`a la nostra applicazione.
                Ad esempio delle normative vigenti in quell ambito (vedi avionico, ferroviario, etc\dots).
    \end{itemize}    
\end{center}
\subsection{Analisi, specifica e validazione dei requisiti}
Nella fase di \textbf{analisi dei requisti }
si stabilscono le funzionalit\`a, i servizi e i vincoli richiesti al sistema.
Il documento prodotto da questa fase \`e il DSR (Documento di specifica dei requisiti) ed 
eventualmente un PTS (piano di testing).

In questa fase si specifica generalmente in linguaggio naturale,
 \`e la fase iniziale, in cui i requisiti sono descritti in modo pi\`u semplice.
 Spesso il documento di analisi \`e svolto da analisti a stretto contatto con gli stakeholders.
\\
Nella fase di \textbf{specifica dei requisiti}, c'\`e un processo di schematizzazione
dei requisiti, spesso in un documento strutturato (template)
\\
Nella fase di \textbf{validazione} invece, si rivedono le richieste spesso con i Clienti
in modo di trovare errori e incongruenze.

\subsubsection{Caratteristiche di un SRS(specifica dei requisiti di sistema)}
le caratterisctiche di un SRS sono:
\begin{enumerate}
    \item corretto: Non deve contentere errori
    \item completo: Deve contenere tutti i requisiti
    \item non ambiguo: I requisiti devono essere esplicitati in modo chiaro senza ambiguit\`a
    \item verificabile 
    \item consistente: I requisiti non devono avere contraddizioni
    \item modificabile: una struttura e uno stile facilemnte modificabile
    \item tracciabile: sia a posteriori che antecedenti, quindi deve essere facile tracciare il codice
            a partire dal requisito e viceversa(spesso si usa una numerazione come nello standard IEEE 830-1998)
\end{enumerate}
